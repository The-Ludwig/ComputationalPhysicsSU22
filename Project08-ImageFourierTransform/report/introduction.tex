\section{Discrete Fourier Transform}

The \emph{Discrete Fourier Transform (DFT)} of a series of complex numbers $x_0, \dots, x_{N-1}$
are the $N$ numbers defined as
\begin{equation}
    X_k = \frac{1}{\sqrt{N}}\sum_{n=0}^{N-1} x_n e^{-i\frac{2\pi}{N}n k}.
\end{equation}
For intuitive understanding one can think of $X_k$ as the intensity of the frequency,
$\nu =\frac{\omega}{2\pi}= k / N$ in the signal.

For a deeper understanding let's take a step back and remember that Fourier
found that a periodic function can be represented as a weighted sum of cosine and
sine functions.
This does not only apply for continuous, but also discontinuous functions,
such as a square wave.
To put it in a more mathematical way: The
$L^2([0, T])$-space\footnote{This means every complex-valued function defined on $[0, T]$, which's absolute square is integrable over $[0, T]$.}
has the orthonormal basis functions
\begin{equation}
    \phi_k(x)=\exp\left({-2ikx\frac{2\pi}{T}}\right) \quad k\in \mathbb{Z},
\end{equation}
so any function $f(x)\in L^{2}([0, T])$ can be written as
\begin{equation}
    f(x) \equiv \sum_{k=-\infty}^{\infty} \underbrace{\left< f, \phi_k \right>}_{a_k} \phi_k(x)
    = \sum_{k=-\infty}^{\infty} {a_k} \exp\left({-2ikx\frac{2\pi}{T}}\right)
\end{equation}
where $\equiv$ means "equals almost everywhere under the norm induced by the scalar product $\left<\cdot,\cdot\right>$".
This sum is called a Fourier series.
The scalar product (and thus the norm) in this space is
\begin{equation}
    \left< f, g\right> = \int_0^T f(x)g^*(x) \dif x.
\end{equation}
Thus, the Fourier coefficients $a_k$ are
\begin{equation}
    a_k = \int_0^T f(x) \exp\left({ikx\frac{2\pi}{T}}\right) \dif x
\end{equation}
This is a very abstract definition, but if we get used to the math,
this only precises the findings of Fourier: A periodic function
(our functions are restricted to $[0, T]$, but they can be thought of as being periodically continued everywhere else)
can be represented as a weighted (infinite) sum of sines and cosines (in our formulation hidden with the Euler identity in $e^{ix}$).

If we now have a function which is neither confined nor periodic, we can handwavely argue, that $T$ goes to infinity and
that we need a continues set of basis functions. This is a way of approaching the continuous \emph{Fourier Transform (FT)}.
The idea of the FT is to transform the original function $f$ to another function $\hat f$ without loosing any information and thus being
able to transform the function back.
The transformation is similar to the definition of the Fourier coefficients
\begin{equation}
    \hat f(\nu) = \int_{-\infty}^{\infty} f(x) e^{-i2\pi\nu x} \dif x.
\end{equation}
The inverse Fourier transform is given by
\begin{equation}
    f(x) = \int_{-\infty}^{\infty} \hat f(\nu) e^{i2\pi\nu x} \dif \nu.
\end{equation}
For this the function $f: \mathbb{R}\to \mathbb{C}$ only has to be absolutely integrable, so that
$\int_{\mathbb{R}} \abs{f(x)} \dif x$ is defined.

The correctness of the inverse-rule can be shown if we express the Dirac $\delta$-function as
\begin{equation}
    \delta(x) = \int_{-\infty}^{\infty} e^{-i2\pi \nu x} \dif \nu.
\end{equation}

Again $\hat f(\nu)$ can be thought of as the intensity of the frequency $\nu$ in the function. That's
why we name the domain $\nu$ and $\hat f$ is often called $f$ in frequency-space.
This definition is easily extended to functions which take multi-dimensional input $f: \mathbb{R}^n \to \mathbb{C}$
\begin{equation}
    \hat f(\vec \nu) = \int_{\mathbb{R}^n} f(\vec x) e^{-i2\pi\vec \nu \cdot \vec x} \dif{}^n x.
\end{equation}
And the inverse transform likewise.

Now we can motivate the DFT: If we define a function (to be exact: a distribution) as
\begin{equation}
    f(x) = \frac{1}{\sqrt N} \sum_{n=0}^{N-1} x_n\ \delta\left(x-{n}\right)
\end{equation}
the DFT of the points is the FT of this function at the frequencies $k/N$
\begin{equation}
    \begin{split}
        \hat f\left(\frac{k}{N}\right) = \int_{-\infty}^{\infty}  \frac{1}{\sqrt N} \sum_{n=0}^{N-1}
        x_n\ \delta\left(x-{n}\right) e^{-i2\pi\frac{k}{N} x} \dif x\\
        = \frac{1}{\sqrt N} \sum_{n=0}^{N-1} x_n e^{-i2\pi k \frac{n}{N}}
        = X_k.
    \end{split}
\end{equation}
And likewise, if we define the function
\begin{equation}
    F(\nu) = \frac{1}{\sqrt N} \sum_{n=0}^{N-1} X_n\ \delta\left(\nu-{n}\right)
\end{equation}
it's inverse FT at $k/N$ is the inverse DFT of the points $X_k$
\begin{equation}
    \begin{split}
        \hat F\left(\frac{k}{N}\right)
        = \int_{-\infty}^{\infty}  \frac{1}{\sqrt N} \sum_{n=0}^{N-1}
        X_n\ \delta\left(\nu-{n}\right) e^{i2\pi\frac{k}{N} \nu} \dif \nu\\
        = \frac{1}{\sqrt N} \sum_{n=0}^{N-1} X_n e^{i2\pi k \frac{n}{N}}
        = \frac{1}{N} \sum_{n,m=0}^{N-1}
        x_m e^{-i2\pi \frac{n}{N}(m-k)}\\
        =  \sum_{m=0}^{N-1}
        x_m\delta_{m,k} = x_k.
    \end{split}
\end{equation}
Where we have used the relation with the geometric series
\begin{equation}
    \frac{1}{N}
    \sum_{n=0}^{N-1}
    e^{-i2\pi \frac{n}{N}(m-k)}=
    \begin{cases}
        \frac{1-e^{-i2\pi (m-k)}}{1-e^{-i2\pi\frac{m-k}{N}}} = 0 & \forall m\neq k \\
        1                                                        & \forall m=k
    \end{cases}
    = \delta_{m,k}.
\end{equation}
The definition of the function might seem abstract at first glance, but it is nothing
more than a more mathematical way of expressing, that we only have $N$ samples of
an unknown functions at these points.
The normalization factor of $1/\sqrt N$ is chosen so the inverse transform does not rescale
the points.

As with the FT, the DFT is easily enlarged to more dimensions
\begin{equation}
    X_{l,m,n} = \frac{1}{\sqrt{N_1N_2N_3}}\sum_{i=0}^{N_1-1}\sum_{j=0}^{N_2-1}\sum_{k=0}^{N_3-1} x_{i,j,k} e^{-i{2\pi}\ \left(\!\frac{li}{N_1}+\frac{mj}{N_2}+\frac{nk}{N_3}\right)}.
\end{equation}
and the inverse
\begin{equation}
    x_{l,m,n} = \frac{1}{\sqrt{N_1N_2N_3}}\sum_{i=0}^{N_1-1}\sum_{j=0}^{N_2-1}\sum_{k=0}^{N_3-1} X_{i,j,k} e^{i{2\pi}\ \left(\!\frac{li}{N_1}+\frac{mj}{N_2}+\frac{nk}{N_3}\right)}.
\end{equation}
I chose the three-dimensional example here, since we are going to look at images with a color channel, which is three dimensional data.
Other dimensionalities are analogous.

Since we will only have real numbers, we can restrict the DFT to real numbers:

\section{The Numerical Implementation: The Fast Fourier Transform}
Implementing directly the definition of the DFT might not be very slow, but we can do extremely better
using the fast Fourier transform (FFT) algorithm.